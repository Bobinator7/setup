%##########################################################################
%                                                                     
%                        Care-O-bot User manual - UR arm section                                                   
%                                     
%##########################################################################


%##########################################################################
% Formatierungsoptionen
% fuer Bilder die Option "draft" entfernen
\documentclass[12pt,twoside]{report}

% Standard Style-Files
%\usepackage{german}
\usepackage{a4}
%\usepackage{psfig}
\usepackage{graphicx}
\usepackage{subfigure}
%\usepackage{equations}
\usepackage{thb}
\usepackage{amssymb}
\usepackage{listings}

% define colors
\usepackage{color}
\definecolor{light-gray}{gray}{0.85}

% Abruerzungsverzeichnis
\usepackage{nomencl}

% Hyperlinks
\usepackage[bookmarksnumbered=true,backref=page,breaklinks=true,pdftitle={Care-O-bot manual},
pdfauthor={Florian Weisshardt},pdfsubject={Care-O-bot administration and user manual},
pdfkeywords={Care-O-bot, ROS, setup, install, manual}]{hyperref}


% Seitenstil
\pagestyle{headings}
%
% Abstand zwischen Abs"atzen
\setlength{\parskip}{1.5ex}

% Einr"uckung der ersten Zeile eines Absatzes unterdr"ucken
\setlength{\parindent}{0pt}

% Grosszuegigere Wortabstaende
\sloppy

% Tiefe der numerierten Kapitel definieren
\setcounter{secnumdepth}{3}

% Tiefe der Kapitel im Inahltsverzeichnis definieren
\setcounter{tocdepth}{2}


% Damit Bilder m"oglichst da sind, wo man sie will
\setcounter{topnumber}{20}
\setcounter{bottomnumber}{20}
\setcounter{totalnumber}{20}
\renewcommand{\topfraction}{.9999}
\renewcommand{\bottomfraction}{.9999}
\renewcommand{\textfraction}{0}

% source code
\lstset{
basicstyle=\footnotesize,
frame=single,
breaklines=true,
backgroundcolor=\color{light-gray}
}


%##########################################################################
% Abkuerzungen 
\let\abbrev\nomenclature
\renewcommand{\nomname}{Abk"urzungsverzeichnis} 
\setlength{\nomlabelwidth}{.24\hsize} % Punkte zw. Abkrzung und Erklrung
\renewcommand{\nomlabel}[1]{#1 \dotfill}
%\setlength{\nomitemsep}{-\parsep} % Zeilenabstnde verkleinern
\makenomenclature 
% einfuegen mit \abbrev{Abkuerzung}{Beschreibung}


%###########################################################################
% Bearbeitung von einzelnen Kapiteln

%\includeonly{berichttitle}
%\includeonly{berichttoc}
%\includeonly{bericht1}
%\includeonly{bericht2}
%\includeonly{bericht3}
%\includeonly{bericht4}
%\includeonly{bericht5}
%\includeonly{berichtapp}
%\includeonly{berichtlof}
%\includeonly{berichtloc}
%\includeonly{berichtlot}
%\includeonly{berichtbib}


%###########################################################################
\begin{document}

%################
%   Titelseite
%################
\begin{titlepage}
\vspace*{13mm}
\begin{center}
  \vspace{10mm} 
         {\large \hspace{20mm} Agenda and Content\\}
  \vspace{10mm}
       {\Large
          \bf
          \hspace{20mm} Introduction workshop\\} 
  \vspace{5mm}
       {\Large
          \bf
          \hspace{20mm} Care-O-bot \\}

  \vspace{80mm}
  \makebox[40mm]{\large \hspace{16mm} Autors: }\makebox[65mm][l]
                                   {\large Florian Weisshardt}
% \makebox[40mm]{}\makebox[65mm][l]{\large Name}\\
  \vspace{10mm}
         {\large \hspace{20mm} Fraunhofer IPA} \\
  \vspace{5mm}
         {\large \hspace{20mm} Institute for Manufacturing Engineering and Automation} \\
         {\large \hspace{20mm} Stuttgart, Germany} \\
  %\vspace{20mm}
  \vfill
         {\large \hspace{20mm} Last modified on \today}
\end{center}
\end{titlepage}

%################
%   intermediate pages
%################
\clearpage
\thispagestyle{empty}
\cleardoublepage
\thispagestyle{empty}\cleardoublepage % Inhalt auf der rechten Seite beginnen
% Raendereinstellungen fuer Doppelseitigen Ausdruck
\evensidemargin=2pt
\oddsidemargin=40pt
% Zeilenabstand strecken
\renewcommand{\baselinestretch}{1.15}\normalsize
\pagenumbering{roman}
\pagenumbering{arabic}

%################
%   begin content
%################
\chapter{Agenda}
This is an Agenda for a two days workshop after shipping the robot. It will cover topics for unpacking, setting up, safety introduction, starting up the robot, first steps for moving the robot, navigation introduction.

\begin{table}[htb]
\begin{tabular}{|l|l|l|}
  \hline
  Topic & approx. Duration & involved persons \\ \hline \hline
  Unpacking the robot & 2h & customer contact person (1 person) \\ \hline
  Technical handover & 0.5h & customer contact person (1 person) \\ \hline
  Presentation: Introduction to ROS and Care-O-bot & 1h & all interessted people (10-30 persons) \\ \hline
  Safety instructions & 1h & customer contact person, robot administrator, all people working with robot (5-15 persons) \\ \hline
  Starting up the robot & 1h & robot administrator, all people working with robot (5-15 persons) \\ \hline
  First steps for moving the robot & 3h & robot administrator, all people working with robot (5-15 persons) \\ \hline
  Important ROS packages & 1h & robot administrator, all people working with robot (5-15 persons) \\ \hline
  Introduction to navigation & 2h & robot administrator, all people working with robot (5-15 persons) \\ \hline
  Introduction to administration & 1h & robot administrator \\ \hline 
  
  
\end{tabular}
\end{table}

\chapter{Content}

\section{Unpacking}
loocking for transportation damage, taking photos

show how to fix the robot in the box

show how to protect the robot in the box from getting scratches and losing parts

content of supply box

\section{Technical handover}
go through the daily morning show

sign a daily morning show protocol (mark damages or errors)

\section{Presentation: Introduction to ROS and Care-O-bot}
self introduction from the robot

\subsection{Introduction to ROS}
See slides from ROS workshop on 1.10.13 in Stuttgart
\subsection{Introduction to Care-O-bot}
See slides from 24.10.13 in Odense: Motivation, Hardware, applications, SW architecture, community, testing, collabration

\section{Safety instructions}
what issues to be taken care about: see slides from 24.10.14 in Odense

show how to stop the robot: buttons, laser canner, wireless emergency stop

show how to release the emergency stop again

charging the robot

\section{Starting up the robot}
turn key, login, run bringup, initialise, diagnostics dashboard: see Care-O-bot manual

\section{First steps for moving the robot}
joystick, command gui

simple\_script\_server: blocking and non-blocking, using predefined positions and direct joint positions, leds, sound

cob\_default\_robot\_config: add your own package for robot configuration, add new predefined positions, add new command gui buttons

cob\_default\_env\_config: add environment specific parameter to your own package, add buttons to command gui

\section{Introduction to navigation}
show various navigation possibilities dwa, tr, linear

visualize and command through rviz

cob\_navigation\_local

cob\_mapping\_slam: tips tricks for creating a map

cob\_navigation\_global

cob\_navigation\_slam

\section{Important ROS packages}
cob\_bringup

cob\_hardware\_config

cob\_calibration\_data

cob\_default\_robot\_config

cob\_default\_env\_config

tf frames

simulation

where to start which ROS node: distribution of CPU and network traffic

\section{Introduction to administration}
pc and network hardware setup: router, pcs, extension cards, CAN, usb, ethernet, camera network, CPU, RAM, disk usage

network configuration: IP adresses, DHCP, DNS, integration into building network

pc configuration on robot: ntp, nfs, robot user, 

setup repository: manual, cobadduser, cob-pcs-install, cob-pcs-execute

overcome wireless emergency stop

ROS configuration: bashrc, ROS\_PACKAGE\_PATH, ROS\_MASTER\_URI, bringup stacks, user overlays


\end{document}
%##########################################################################
