%##########################################################################
%                                                                     
%                        Care-O-bot User manual - UR arm section                                                   
%                                     
%##########################################################################


%##########################################################################
% Formatierungsoptionen
% fuer Bilder die Option "draft" entfernen
\documentclass[12pt,twoside]{report}

% Standard Style-Files
%\usepackage{german}
\usepackage{a4}
%\usepackage{psfig}
\usepackage{graphicx}
\usepackage{subfigure}
%\usepackage{equations}
\usepackage{thb}
\usepackage{amssymb}
\usepackage{listings}

% define colors
\usepackage{color}
\definecolor{light-gray}{gray}{0.85}

% Abruerzungsverzeichnis
\usepackage{nomencl}

% Hyperlinks
\usepackage[bookmarksnumbered=true,backref=page,breaklinks=true,pdftitle={Care-O-bot manual},
pdfauthor={Nadia Hammoudeh Garcia, Florian Weisshardt},pdfsubject={Care-O-bot administration and user manual},
pdfkeywords={Care-O-bot, ROS, setup, install, manual}]{hyperref}


% Seitenstil
\pagestyle{headings}
%
% Abstand zwischen Abs"atzen
\setlength{\parskip}{1.5ex}

% Einr"uckung der ersten Zeile eines Absatzes unterdr"ucken
\setlength{\parindent}{0pt}

% Grosszuegigere Wortabstaende
\sloppy

% Tiefe der numerierten Kapitel definieren
\setcounter{secnumdepth}{3}

% Tiefe der Kapitel im Inahltsverzeichnis definieren
\setcounter{tocdepth}{2}


% Damit Bilder m"oglichst da sind, wo man sie will
\setcounter{topnumber}{20}
\setcounter{bottomnumber}{20}
\setcounter{totalnumber}{20}
\renewcommand{\topfraction}{.9999}
\renewcommand{\bottomfraction}{.9999}
\renewcommand{\textfraction}{0}

% source code
\lstset{
basicstyle=\footnotesize,
frame=single,
breaklines=true,
backgroundcolor=\color{light-gray}
}


%##########################################################################
% Abkuerzungen 
\let\abbrev\nomenclature
\renewcommand{\nomname}{Abk"urzungsverzeichnis} 
\setlength{\nomlabelwidth}{.24\hsize} % Punkte zw. Abkrzung und Erklrung
\renewcommand{\nomlabel}[1]{#1 \dotfill}
%\setlength{\nomitemsep}{-\parsep} % Zeilenabstnde verkleinern
\makenomenclature 
% einfuegen mit \abbrev{Abkuerzung}{Beschreibung}


%###########################################################################
% Bearbeitung von einzelnen Kapiteln

%\includeonly{berichttitle}
%\includeonly{berichttoc}
%\includeonly{bericht1}
%\includeonly{bericht2}
%\includeonly{bericht3}
%\includeonly{bericht4}
%\includeonly{bericht5}
%\includeonly{berichtapp}
%\includeonly{berichtlof}
%\includeonly{berichtloc}
%\includeonly{berichtlot}
%\includeonly{berichtbib}


%###########################################################################
\begin{document}

% Titelseite einfgen
%###########################################################################
%
%   Titelseite
%
%###########################################################################
\begin{titlepage}
\vspace*{13mm}
\begin{center}
  \vspace{10mm} 
         {\large \hspace{20mm} Care-O-bot Manual\\}
  \vspace{10mm}
       {\Large
          \bf
          \hspace{20mm} Extension for\\} 
  \vspace{5mm}
       {\Large
          \bf
          \hspace{20mm} Universal Robot UR5 and connector\\}

  \vspace{80mm}
  \makebox[40mm]{\large \hspace{16mm} Autors: }\makebox[65mm][l]
                                   {\large Florian Weisshardt}
  \makebox[40mm]{}\makebox[65mm][l]{\large Nadia Hammoudeh Garcia}\\
  \makebox[40mm]{}\makebox[65mm][l]{\large Bernhard Waterkamp}\\
% \makebox[40mm]{}\makebox[65mm][l]{\large Name}\\
  \vspace{10mm}
         {\large \hspace{20mm} Fraunhofer IPA} \\
  \vspace{5mm}
         {\large \hspace{20mm} Institute for Manufacturing Engineering and Automation} \\
         {\large \hspace{20mm} Stuttgart, Germany} \\
  %\vspace{20mm}
  \vfill
         {\large \hspace{20mm} Last modified on \today}
\end{center}
\end{titlepage}

\clearpage
\thispagestyle{empty}
\cleardoublepage

\thispagestyle{empty}\cleardoublepage % Inhalt auf der rechten Seite beginnen

% Raendereinstellungen fuer Doppelseitigen Ausdruck
\evensidemargin=2pt
\oddsidemargin=40pt

% Zeilenabstand strecken
\renewcommand{\baselinestretch}{1.15}\normalsize


\pagenumbering{roman}

% Inhaltsverzeichnis einfgen
%\tableofcontents
%\thispagestyle{empty}\cleardoublepage

% Abkrzungsverzeichnis einfgen
%\include{berichtnom}
%\thispagestyle{empty}\cleardoublepage

% Kapitel einfgen
\pagenumbering{arabic}
\chapter{Universal Robot on Care-O-bot}
This chapter is an addition to the Care-O-bot manual which can be found at \url{https://github.com/ipa320/setup/raw/master/manual/Care-O-bot_manual.pdf} and explains handling the Universal Robot UR5 arm on Care-O-bot.








\end{document}
%##########################################################################
