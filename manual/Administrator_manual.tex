\chapter{Administrator manual}
\label{chap:admin}

%#################################################################################################
\section{Setup robot pcs}
%#################################################################################################
On all Care-O-bots there are at least two pcs. Some Care-O-bots have an optional third pc, which is not coveres by this manual. Within this section we will guide you through setting up new pcs. When nothing otherwise is mentioned the following instructions are for both pc1 and pc2, please do the same steps on both pcs.

To pc1 all actuators are connected, sensors are connected both, to pc1 and pc2. All camera sensors are connected to pc2, whereas all other sensors like e.g. laser scanners are connected to pc1. By default pc3 is not connected to any hardware and therefore can be used as additional computing power.

\subsection{Install operating system}
The first step is to install the operating system for each pc, which means pc1 and pc2 (optionally pc3). We are using Ubuntu as the main operating system for the robot. We recommend to install the \textbf{Ubuntu 10.4 LTS (long term stable) 64-bit} version because this version is well tested to work with the hardware. 

For the first installating please install Ubuntu (english version) creating a normal swap partition. Please choose \textit{robot} as an admin account with a really safe password which should only be known to the local robot administrator. The hostname of the pc should be \textit{cob3-X-pc1} and \textit{cob3-X-pc2}.

\subsection{Install basic tools}
Next we have to install some basic tools for the further setup of the pcs. In order to install the packages a internet connection is needed.

\begin{lstlisting}
sudo apt-get update
sudo apt-get install vim tree openssh-server gitg meld curl
\end{lstlisting}

To facilitate the further setup we created a setup repository with some helpfull scripts. To checkout the setup repository use:

\begin{lstlisting}
mkdir ~/git
cd ~/git
git clone git://github.com/ipa320/setup.git
\end{lstlisting}

\subsection{Setup internal robot network}
Inside the robot there's a router which connects the pcs and acts as gateway to the building network. Setup the router with the following configuration. 

The ip adress of the router should be \textbf{192.168.0.1} and for the internal network dhcp should be activated. Use \textbf{cob3-X} as hostname for the router. Register the MAC adresses of pc1 and pc2 so that they get a fixed ip adress over dhcp. Use \textbf{192.168.0.101 for pc1} and \textbf{192.168.0.102 for pc2}. Enable \textbf{portforwarding} for port 2201 to 192.168.0.101 and for port 2202 to 192.168.0.102.

After ensuring that the network configuration of the router is setup correctly, we can configure the pcs. All pcs should have two ethernet ports. The upper one should be connected to the internal router. Sometimes the graphical network manager causes troubles, so it is best to remove it

\begin{lstlisting}
sudo apt-get remove network-manager
\end{lstlisting}

After removing the network manager we will have to edit \textit{/etc/network/interfaces} manually. 

\subsubsection{Network configuration on pc1}

\begin{lstlisting}
auto lo
iface lo inet loopback

auto eth0
iface eth0 inet static
address 192.168.0.101 # internal ip adress of pc1
netmask 255.255.255.0 # netmask

auto eth1
iface eth1 inet static
address 192.168.42.1 # ip adress for controller network
netmask 255.255.255.0 # netmask
\end{lstlisting}

\subsubsection{Network configuration on pc2}

\begin{lstlisting} 
auto lo
iface lo inet loopback

auto eth0
iface eth0 inet static
address 192.168.0.102 # internal ip adress of pc2
netmask 255.255.255.0 # netmask

auto eth1
iface eth1 inet static
address 192.168.21.99 # ip adress for camera network
netmask 255.255.255.0 # netmask
\end{lstlisting}

\subsection{Install NFS}
After the network is configured properly we can setup a NFS netween the robot pcs. pc2 will act as the NFS server and pc1 as NFS client.

\subsubsection{NFS configuration on pc2 (server)}
Install the NFS server package and create the NFS directory

\begin{lstlisting}
sudo apt-get install nfs-kernel-server
sudo mkdir /u 
\end{lstlisting}

Add the following following line to \textit{/etc/fstab}:

\begin{lstlisting}
/home	/u	none	bind	0	0
\end{lstlisting}

Now we can mount the drive

\begin{lstlisting}
sudo mount /u
\end{lstlisting}

Activate IDMAPD in \textit{/etc/default/nfs-common} by changing the NEED\_IDMAPD to yes

\begin{lstlisting}
NEED_IDMAPD=yes
\end{lstlisting}

Copy the file \textit{$\sim$/git/setup/nfs\_setup/server/exports} to \textit{/etc/exports}

\begin{lstlisting}
cp ~/git/setup/nfs_setup/server/exports /etc/exports
\end{lstlisting}

Change the home directory of the \textit{robot} user from \textit{/home/username} to \textit{/u/username} in the \textit{/etc/passwd} file.

After finishing you need to reboot the pc

\begin{lstlisting}
sudo reboot
\end{lstlisting}

\subsubsection{NFS configuration on pc1 (client)}
Install the NFS client package and create the NFS directory

\begin{lstlisting}
sudo apt-get install nfs-kernel-server autofs
sudo mkdir /u
\end{lstlisting}

Activate IDMAPD in \textit{/etc/default/nfs-common} by changing the NEED\_IDMAPD to yes

\begin{lstlisting} 
NEED_IDMAPD=yes
\end{lstlisting}

Edit \textit{/etc/auto.master} and add

\begin{lstlisting}
/-	/etc/auto.direct
\end{lstlisting}

Copy the file \textit{$\sim$/git/setup/nfs\_setup/client/auto.direct} to \textit{/etc/auto.direct}

\begin{lstlisting}
cp ~/git/setup/nfs_setup/client/auto.direct /etc/auto.direct
\end{lstlisting}

Activate the NFS

\begin{lstlisting}
sudo update-rc.d autofs defaults
sudo service autofs restart
sudo modprobe nfs
\end{lstlisting}

Change the home directory of the \textit{robot} user from \textit{/home/username} to \textit{/u/username} in the \textit{/etc/passwd} file.

After finishing you need to reboot the pc

\begin{lstlisting} 
sudo reboot
\end{lstlisting}

\subsection{Setup NTP time synchronitation}
Install the ntp package

\begin{lstlisting}
sudo apt-get install ntp
\end{lstlisting}

\subsubsection{NTP configuration on pc1 (NTP server)}
Edit \textit{/etc/ntp.conf}, change the server to \textit{0.pool.ntp.org} and add the restrict line

\begin{lstlisting} 
server 0.pool.ntp.org
restrict 192.168.0.0 mask 255.255.255.0 nomodify notrap
\end{lstlisting}

\subsubsection{NTP configuration on pc2 (NTP client)}
Edit \textit{/etc/ntp.conf}, change the server to \textit{cob3-X-pc1}:
\begin{lstlisting}
server cob3-X-pc1
\end{lstlisting}

\subsection{Install ROS and Care-O-bot driver software}
For general instructions see \url{http://www.ros.org/wiki/Robots/Care-O-bot/electric}.

\subsubsection{Install ROS and additional tools}
\begin{lstlisting}
sudo apt-get install openjdk-6-jdk zsh terminator
sudo apt-get install python-setuptools
sudo easy_install -U rosinstall
sudo apt-get install ros-diamondback-care-o-bot ros-diamondback-perception-pcl-addons ros-diamondback-erratic-robot
sudo apt-get install ros-electric-care-o-bot ros-electric-perception-pcl-addons ros-electric-pr2-desktop ros-electric-pr2-robot ros-electric-pr2-apps pr2-power-drivers 
\end{lstlisting}

\subsubsection{Setup bash environment}
We setup a special bash environment to be used on the Care-O-bot pcs. The environments differ from pc1, pc2 and pc3. Copy the \textit{cob-bash-bashrc.pcX} to \textit{/etc/cob-bash-bashrc} on each pc.

\begin{lstlisting}
sudo cp ~/git/setup/cob-pcs/cob-bash-bashrc.pcX /etc/cob-bash-bashrc
\end{lstlisting}

All users have a pre-configured bash environment too, therefore copy \textit{user.bashrc} to \textit{$\sim$/.bashrc}
\begin{lstlisting}
cp ~/git/setup/cob-pcs/user.bashrc ~/.bashrc
\end{lstlisting}

If you logout and login again or source your \textit{$\sim$/.bashrc}, you should see different terminal colors for each pc and the \textit{ROS\_PACKAGE\_PATH} should be configured. If you check 
\begin{lstlisting}
roscd cob_bringup
\end{lstlisting}
you should end up in \textit{/u/robot/git/care-o-bot/cob\_robots/cob\_bringup}

\subsubsection{Setup hardware components}
In order to use the different hardware components we have to install the drivers and set permission rights. All hardware configuration is stored in the \textit{cob\_hardware\_config} package.

\paragraph{Setup udev rules}
TIn order to have fixed device names we setup udev rules for Care-O-bot. Copy the udev rules from the setup repository to \textit{/etc/udev/rules.d} on pc1:
\begin{lstlisting}
sudo cp ~/git/setup/udev_rules/01-cob.rules /etc/udev/rules.d
\end{lstlisting}

\paragraph{Sick S300 laser scanners}
The Sick S300 scanners on the frontside and backside of the robot are connected via USB to pc1. To recevice data from the Sick S300 scanners check if the user is in the \textit{dialout} group
\begin{lstlisting}
groups
\end{lstlisting}

For testing you can run the front laser scanner with
\begin{lstlisting}
roslaunch cob_bringup laser_front.launch
\end{lstlisting}

To check if there is some data published use
\begin{lstlisting}
rostopic hz /scan_front
\end{lstlisting}

Check the rear scanner in the same way.

tbd, setup safety region

\paragraph{Hokuyo URG laser scanner}
tbd

\paragraph{Relayboard}
tbd

\paragraph{Base}
tbd, configure elmo controllers

\paragraph{Tray sensors}
tbd, phidget

\paragraph{Schunk SDH with tactile sensors}
tbd, firmware version
Add the robot user to the group dialout in the /etc/group file.

\paragraph{Schunk powercubes}
tbd, pcan

\paragraph{Head axis}
tbd

\paragraph{Prosilica cameras}
You have to define the ip address for the cameras, on pc2 in the file /etc/network/interfaces you can copy these lines:
\begin{lstlisting}
auto eth1 
iface eth1 inet static
address 192.168.21.99   # IP of network adapter to cameras 
netmask 255.255.255.0   # netmask
\end{lstlisting}

\paragraph{Kinect}
pc2


\subsection{Create a new user account}
\label{sec:account}
Due to the fact that all users need to be in the correct user groups, that the bash environment needs so setup correctly and that user ids need to be synchonised between all pcs for the NFS to work, we facilitate the creation of a new user with a \textit{cobadduser} script. On pc2 and with administration rights you can add a new user with the following instruction
\begin{lstlisting}
cd ~/git/setup
sudo ./cobadduser new_user_name
\end{lstlisting}

All you need to do now is to login as the new user, create an ssh-key and add the key to the authorized keys to be able to login to all pcs without password. This is neccesarry for launching nodes remotely on all pcs.
\begin{lstlisting}
ssh-keygen
ssh-copy-id cob3-X-pc1
\end{lstlisting}


\section{Network}
Make sure you have name resolution and access to the robot pcs from your external pc. To satisfy the ROS communication you need a full DNS name lockup for all machines. Check it from your remote pc with
\begin{lstlisting}
ping 192.168.0.101
ping cob3-X-pc1
\end{lstlisting}
and the other way round try to ping your remote pc from one of the robot pcs
\begin{lstlisting}
ping your_ip_adress
ping your_hostname
\end{lstlisting}

If ping and DNS is not setup correctly, there are multiple ways to enable access and name resolution.

\subsection{Setting up your building network}
Setting up you building network to enable dns and portforwarding to the internal network.

\subsection{Manual setup for each remote pc}
You can setup a route to the internal network addresses. Please change the robot name and your network device to fit your settings. E.g. for connecting to cob3-X on eth0
\begin{lstlisting}
sudo route add -net 192.168.0.0 netmask 255.255.0.0 gw cob3-X dev eth0
\end{lstlisting}

for connecting to cob3-X on wlan0
\begin{lstlisting}
sudo route add -net 192.168.0.0 netmask 255.255.0.0 gw cob3-X dev wlan0
\end{lstlisting}


For name resolution you will probably have to edit the file \textit{/etc/hosts} on the robot pcs as well as on the remote pc. Therefore add the following addresses to the \textit{/etc/hosts} of your remote pc.
\begin{lstlisting}
192.168.0.101 cob3-X-pc1
192.168.0.102 cob3-X-pc2
192.168.0.103 cob3-X-pc3
\end{lstlisting}

Add your ip adress and hostname to the \textit{/etc/hosts} of all robot pcs.

\section{Calibration}
Now is not working....
\section{Backup and restoring users}   
Now is not working...    


