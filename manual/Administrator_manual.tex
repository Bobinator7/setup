\chapter{Administrator manual}   

\section{Setup robot pcs}

\subsection{install operating system}
ubuntu 10.4 64-bit

user password


\subsection{Install basic tools}
  \colorbox{light-gray}{
         \begin{minipage}{1.0\textwidth} 
	sudo apt-get install vim tree openssh-server ntp
         \end{minipage} }


You can use the instructions of the setup repository, to obtain it you can use git: 
\\
 \colorbox{light-gray} {
         \begin{minipage}{1.0\textwidth} 
  	sudo apt-get update \\ 
	sudo apt-get git-core \\
	git clone git@github.com:ipa320/setup.git
         \end{minipage} 
         }  \\
\\
\subsection{Install nfs}

\subsubsection{Install server (pc2)}

%\\ {\bf On server (pc2): }
   \colorbox{light-gray}{
         \begin{minipage}{1.0\textwidth} 
	sudo apt-get install nfs-kernel-server \\
	sudo mkdir /u 
         \end{minipage} 
         } \\
%\\
\\ Add the following following line to /etc/fstab: \\
   \colorbox{light-gray}{
         \begin{minipage}{1.0\textwidth} 
		/home	/u	none	bind	0	0
         \end{minipage}  
         } \\
      \\   
   \colorbox{light-gray}{
         \begin{minipage}{1.0\textwidth} 
		sudo mount /u
         \end{minipage}  } \\
\\Activate IDMAPD in /etc/default/nfs-common by changing the NEED\_IDMAPD to yes (\colorbox{cyan}{NEED\_IDMAPD=yes}):\\
\\   \colorbox{light-gray}{
         \begin{minipage}{1.0\textwidth} 
		sudo vi /etc/default/nfs-common
         \end{minipage}  } \\
\\Copy the file setup/nfs\_setup/server/exports to /etc/exports\\
\\Change the home directory of all (cob)-users from /home/username to /u/username in the /etc/passwd file\\
\\Reboot the pc:
\\
   \colorbox{light-gray}{
         \begin{minipage}{1.0\textwidth} 
		sudo reboot
         \end{minipage}  } \\
\\
\subsubsection{Install clients (Pc1)} 
%\\ {\bf On client (pc1 and pc3):}\\
%\\
   \colorbox{light-gray}{
         \begin{minipage}{1.0\textwidth} 
		sudo apt-get install nfs-kernel-server autofs \\
		sudo mkdir /u
        \end{minipage}  } \\
Activate IDMAPD in /etc/default/nfs-common by changing the NEED\_IDMAPD to yes: \\
   \colorbox{light-gray}{
         \begin{minipage}{1.0\textwidth} 
		sudo vi /etc/default/nfs-common
         \end{minipage}  } \\
\\Add the following  line: " \colorbox{cyan}{/-      /etc/auto.direct}" to  /etc/auto.master:\\
\\   \colorbox{light-gray}{
         \begin{minipage}{1.0\textwidth} 
		sudo vi /etc/auto.master
         \end{minipage}  } \\
	\\
Copy file client/auto.direct to /etc/auto.direct and: \\
\\
   \colorbox{light-gray}{
         \begin{minipage}{1.0\textwidth}
	sudo update-rc.d autofs defaults\\
	sudo service autofs restart	\\
	sudo modprobe nfs
         \end{minipage}  } \\
\\
\\Change the home directory of all (cob)-users from /home/username to /u/username in the /etc/passwd file.\\
\\Restart pc :
\\
   \colorbox{light-gray}{
         \begin{minipage}{1.0\textwidth} 
		sudo reboot
         \end{minipage}  } \\
         \\          
         
%{\bf Install ros and Care-O-bot: } \\
\subsection{Install Ros and Care-O-bot driver Software}
You can see how to do it in http://www.ros.org/wiki/Robots/Care-O-bot/electric \\
\\

{\bf Install additional tools:}\\
\\
   \colorbox{light-gray}{
         \begin{minipage}{1.0\textwidth} 
		sudo apt-get install gitg meld curl openjdk-6-jdk zsh terminator\\
		sudo apt-get install python-setuptools\\
		sudo easy\_install -U rosinstall\\
		sudo apt-get install ros-diamondback-care-o-bot ros-diamondback-perception-pcl-addons ros-diamondback-erratic-robot\\
		sudo apt-get install ros-electric-care-o-bot ros-electric-perception-pcl-addons ros-electric-pr2-desktop ros-electric-pr2-robot ros-electric-pr2-apps pr2-power-drivers
         \end{minipage}  } \\

setup ntp time synchronisation
\\
\\
\subsection{Ntp Synchronitation}
{ \bf Edit the ntp.conf change the server to cob3-X-pc1:}\\
\\   \colorbox{light-gray}{
         \begin{minipage}{1.0\textwidth} 
		sudo vi /etc/ntp.conf 
         \end{minipage}  } \\
	\\
\\
{ \bf Setup bash environment:} \\
\\For the three PC's you have to copy cob-bash-bashrc.pcX to /etc/cob-bash-bashrc and for all the users copy user.bashrc to \ /.bashrc you have these files on the setup folder.

\section{PC's Overview}
In PC1 usually bringup the components of the robot, PC2 is used to run the cameras and visual sensors and PC3 can be used to run extra nodes.
When you launch the robot with the bringup file you have these nodes:
\begin{itemize}
\item PC1:
\\ Send the robot\_description to the param server
\\ Start the the robot\_state\_publisher
\\Startup the Hardware , launch the components
\\ Diagnostics
\\ Teleop
\\ Sounds
 \item PC2:
\\ Cameras (left, right, kinects)
\end{itemize}
\section{Network} 
\subsection{Using a route}
You can setup a route to the internal network addresses. Please change the robot name and your network device to t fit your settings. E.g. for connecting to:
\begin{itemize}
\item cob3-X on eth0
\\
\\   \colorbox{light-gray}{
         \begin{minipage}{1.0\textwidth} 
		sudo route add -net 192.168.0.0 netmask 255.255.0.0 gw cob3-X dev eth0
         \end{minipage}  } \\
	\\

\item or cob3-X on wlan0
\\
\\   \colorbox{light-gray}{
         \begin{minipage}{1.0\textwidth} 
		sudo route add -net 192.168.0.0 netmask 255.255.0.0 gw cob3-X dev wlan0
         \end{minipage}  } \\
	\\

\end{itemize}
You can check the settings with: \\
\\   \colorbox{light-gray}{
         \begin{minipage}{1.0\textwidth} 
		ping 192.168.0.101
         \end{minipage}  } \\
	\\

\subsection{Setup name resolution} 
To satisfy the ROS communication you need a full DNS name lockup for all machines. Therefore add the following addresses to your /etc/hosts. Please change the robot name to fit your settings 
\\
\\ \colorbox{light-gray}{
         \begin{minipage}{1.0\textwidth} 
		192.168.0.101 cob3-X-pc1\\
		192.168.0.102 cob3-X-pc2\\
		192.168.0.103 cob3-X-pc3
         \end{minipage} }

You can check the settings with:
\\
\\   \colorbox{light-gray}{
         \begin{minipage}{1.0\textwidth} 
		ping cob3-X-pc1
         \end{minipage}  } \\


\section{Getting an account}
On PC2 and with administration rights you can add an user with the following instruction:
\\
\\   \colorbox{light-gray}{
         \begin{minipage}{1.0\textwidth} 
		sudo adduser new\_user\_name
         \end{minipage}  } 
	\\
	\\
If you want that this new user have also sudo rights you have to add it to the admin group in all PCs.

\section{Calibration}
Now is not working....
\section{Backup and restoring users}   
Now is not working...    

\section{todo}
\begin{itemize}
\item {udev from git\/setup\/udev\_rules\/01-cob.rules copy to \/etc\/udev\/rules.d on pc1}

to check it: ls -l /dev/ and you should have these lines

lrwxrwxrwx 1 root root           7 2012-01-17 10:27 ttyRelais -> ttyUSB0\\
lrwxrwxrwx 1 root root           7 2012-01-17 10:27 ttyScan0 -> ttyUSB1\\
lrwxrwxrwx 1 root root           7 2012-01-17 10:27 ttyScan1 -> ttyUSB2\\
lrwxrwxrwx 1 root root           7 2012-01-17 10:27 ttyTact -> ttyUSB3\\



\item {Camera config you have to change the ip adrrees in the (eth1?) in /etc/network/interfaces it should be on pc2}


auto eth3
iface eth3 inet static
address 192.168.21.99   \# IP of network adapter to cameras
netmask 255.255.255.0   \# netmask

\end{itemize}