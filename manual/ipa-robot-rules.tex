\chapter{IPA robot rules}\label{chap:robot-rules}
You can always get the latest version of this manual at \footnote{\url{https://github.com/ipa320/setup/blob/master/manual/IPA_manual.pdf}}.

If you have comments, suggestions or would like to add something to the manual, please contact \href{mailto:fmw@ipa.fhg.de}{fmw@ipa.fhg.de}.

%##########################################################
%##########################################################
\section{Robot cob3-3 and cob3-5}

%##########################################################
\subsection{Working with the robots}

\subsubsection{Allocate a robot}
Allocate a robot for you, see \footnote{\url{http://care-o-bot.org/trac/wiki/WorkingWithIPARobots}}. 

\subsection{Pay attention to the safety and usage information}
For safety and usage information, see Care-O-bot manual\footnote{\url{https://github.com/ipa320/setup/raw/master/manual/Care-O-bot_manual.pdf}}.

%##########################################################
\subsection{Having a break}
A break is considered a short interruption, e.g. for lunch time or toilet break, where the robot is still allocated for you and you will continue working with the robot afterwards.

\subsubsection{Press the emergency button}
Whenever you are out of reach of the emergency stop, e.g. leaving the room, press one of the emergency buttons.

%##########################################################
\subsection{Leaving the robot}

\subsubsection{Mount all casings}
Mount all casings of the robot which means green and grey parts from base and head casing.

\subsubsection{Move components in home position}
When finishing your work with the robot, leave the robot components in their default position. The default position is
\begin{itemize}
\item Arm in "folded" position
\item Torso in "home" position
\item Tray in "down" position
\item Hand in "home" position
\end{itemize}

TODO: ADD IMAGE

\subsubsection{Move robot to charging station}
When finishing your work with the robot, leave it next to the charging station which is the "home" position for the navigation.

TODO: ADD IMAGE

\subsubsection{Disconnect power}
Unplug the power cable and shut down the power supply.

\subsubsection{Charge the remote emergency stop}
When finishing your work with the robot, put the remote emergency stop into its charging station.

%##########################################################
\section{Robot raw3-1}

\subsection{Leaving the robot}

disconnect power